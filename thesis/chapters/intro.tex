% !TeX root = ../main.tex

\chapter{绪论}

\section{编译器设计竞赛简介}

\subsection{发起背景}

编译原理一直以来都是计算机科学的重点研究方向和热门话题,为检验人才培养成效,计算机类教指委和系统能力培养专家组于2020年共同发起首届 “全国大学生计算机系统能力大赛编译系统设计赛(华为毕昇杯)”。该赛事面向全国高校本科生,以鼓励学生设计、实现一个综合性的编译系统,展示面向特定目标平台的编译器构造与编译优化的能力为目标,提升参赛者在计算机系统设计、分析、优化、应用方面的能力,是我国高校编译系统领域唯一的学科竞赛。

编译系统设计赛要求各参赛队综合运用各种知识(包括但不局限于编译技术、操作系统、计算机体系结构等),构思并实现一个综合性的编译系统,以展示面向特定目标平台的编译器构造与编译优化的能力。
本文就着重对这次比赛由来自中国科学技术大学的队伍“燃烧我的编译器”队的特等奖项目进行了分析和总结,挖掘和研究了项目中的亮点之处。

\subsection{编译器面向语言SysY简介}

SysY 语言是本次大赛要实现的编程语言,是 C 语言的一个子集。每个 SysY程序的源码存储在一个扩展名为sy的文件中。该文件中有且仅有一个名为main的主函数定义,还可以包含若干全局变量声明、常量声明和其他函数定义。SysY语言支持int类型和元素为int类型且按行优先存储的多维数组类型,其中int型整数为32位有符号数;const修饰符用于声明常量。

SysY 语言本身没有提供输入/输出(I/O)的语言构造,I/O是以运行时库方式提供,库函数可以在SysY程序中的函数内调用。

函数:函数可以带参数也可以不带参数,参数的类型可以是int或者数组类型;函数可以返回 int 类型的值,或者不返回值(即声明为 void 类型)。当参数为int时,按值传递;而参数为数组类型时,实际传递的是数组的起始地址,并且形参只有第一维的长度可以空缺。函数体由若干变量声明和语句组成。

变量/常量声明:可以在一个变量/常量声明语句中声明多个变量或常量,声明时可以带初始化表达式。所有变量/常量要求先定义再使用。在函数外声明的为全局变量/常量,在函数内声明的为局部变量/常量。

语句:语句包括赋值语句、表达式语句(表达式可以为空)、语句块、if 语句、while 语句、break语句、continue语句、return语句。语句块中可以包含若干变量声明和语句。

表达式:支持基本的算术运算、关系运算和逻辑运算,真假的表示和界定,关系或逻辑运算的结果,算符的优先级和和结合性以及计算规则(含逻辑运算的“短路计算”)均与 C 语言一致。

\section{论文和项目简要介绍}

\subsection{项目简介}

来自中国科学技术大学的队伍设计的项目“燃烧我的编译器”(下称,本项目)在本次比赛中取得了特等奖的\href{https://compiler.educg.net/2020CSDC}{最好成绩}。该项目已经在Github上开源,网址:\href{https://github.com/mlzeng/CSC2020-USTC-FlammingMyCompiler/}{https://github.com/mlzeng/CSC2020-USTC-FlammingMyCompiler/}。

项目设计上采用类LLVM的三段式编译器架构但别出新意,同时又有很多创新点,且易于阅读学习,开发时支持详细的自动化开发功能验证文件,并能生成详细的汇编代码注释。此外,本项目的拓展性较好,得益于使用了与机器或特定语言无关的中间表示形式IR来组织源代码解析的结果,可以方便的设计成C或其它语言的编译器,并利于迁移到其他常见架构平台。

最关键的是,本项目具有极强的优化能力,作为一个学习用的编译器,在针对SysY语言的大多数比赛测试用例时,在目标机树莓派4B上拥有超过GCC -O3的优化能力。除了简单编译器的经典优化如死代码消除,循环不变量外提,函数内联等优化,本项目还充分利用了树莓派ARM Cortex A7的4核处理器,设计并独创了创新多线程框架,可以更好的降低寄存器切换的开销,使得项目性能在决赛中脱颖而出。

\subsection{项目架构}

本项目采用了类似LLVM的三段式编译器架构,而与LLVM不同之处在于,本项目对中端做了进一步的拓展,创新性的设计了从源代码到机器码的三层IR,并设置三层虚拟指令,细分了后端表示中指令接近物理机的底层程度,下面将对整个项目的架构进行简要说明。

\begin{figure}[htb]
  \centering
  \includegraphics[width=0.8\textwidth]{figures/flammingmycompiler.pdf}
  \caption{本项目的执行流程示意图}
  \label{fig:compiler}
\end{figure}

首先,前端读取SysY源代码,使用Flex进行词法分析,生成token流,在这一步中,编译器主要是识别特殊符号,字符串,关键字和标识符等。然后进行语法分析,将token流的形式组织成抽象语法树AST。最后前端需要采用访问者模式遍历抽象语法树,生成与机器无关的中间表示,即生成IR,严格意义上的IR与特定的编程语言和机器架构实现都无关,这样便于拓展和移植编译器项目。

但是为降低从AST得到IR的难度,即尽量使得IR描述能力接近高级编程语言。同时,后端又希望IR可以方便的转化为机器码,即希望IR能够方便的翻译到底层汇编语言。为了联系这些看起来非常难同时实现的要求,计算机科学领域的其他实现已经告诉我们了答案:分层。本项目采用了三层IR的设计(High IR、Middle IR、Low IR),使用HIR弥补通用IR和AST的差距,使用LIR缩小IR和asm的差距。第二章中会简要介绍本项目的三层IR机制。

项目的后端负责IR到机器码的生成,结合ARM汇编的指令,负责寄存器分配和硬件资源的进一步利用,与中间代码优化的低层 IR Pass 相配合共同完成指令融合、调度和选择等优化。后端部分包含三个层次的虚拟指令,上层指令会选择最小代价智能翻译成一系列下层指令,在第四章会简要介绍。



\subsection{论文组织}

本文的剩余内容主要对上方列举的编译器中端的IR Pass中挑选几个具有代表意义的优化进行着重的理论说明和举例分析,同时兼顾后端中重要的组成部分寄存器分配算法。

第二章将介绍前端产生的IR组织方式,简要介绍由解析器生成的AST形式如何,怎样转换到中间表示,及中间表示的三层组织,并会对HIR的结构作一个简要介绍,将其与AST进行对比来说明HIR为从AST到真正IR所做的铺垫。第三章将介绍中端IR的Pass概念,即对如何对IR做优化,并举了简化控制流程图的例子来具体说明。第四章会介绍本项目中用到的IR的重要表示形式:静态单赋值形式及其构造,会介绍图论中的重要概念——支配理论。第五章介绍基于循环的优化,并举例说明之前构造的SSA对优化的设计有什么好处和简化。第六章介绍后端部分,主要是一些指令的底层化和寄存器分配算法。最后一章是对全文的总结。
\chapter{总结}

在本文中,简要的对编译器中的中间表示的组织方式,经典优化和由IR向机器码的翻译及与指令集架构密切相关的优化等话题做了简要的介绍和举例说明。在笔者进行毕业设计的两个月里,深感自己曾经在课堂上所学过的编译原理知识只是入门性质的介绍,距离实际的现代编译原理和高性能编译器设计还有着很大的差距,因此笔者参考了很多国内外研究生课堂的课件和教材,同时笔者也复习了图论中的很多经典算法,发现计算机科学与离散数学有着千丝万缕的密切联系,最后还认识到,现代的程序员不仅要理解编译器,还要理解计算机的体系结构,只有这样才能写出充分利用硬件性能的好程序。

由于篇幅时间有限,而且笔者才学疏浅,因此文中存在说明不当或者有误之处在所难免,但是本文可以作为后续的本科编译原理教学实验内容的一个简单参考文档,或者作为一个编译器项目的学习笔记,可以为简单但功能俱全且性能优异的编译器设计提供参考思路。

再次对我的指导老师徐伟老师表示诚挚的感谢,徐老师在毕业设计期间为我们提供了很多的专业指导与参考资料。不仅在学术上如此,徐老师还关心帮助学生的生活和学业方向,是位非常好的老师。